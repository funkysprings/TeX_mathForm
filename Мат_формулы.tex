\documentclass[11pt]{article}
\usepackage[T2A]{fontenc}
\usepackage[utf8]{inputenc}
\usepackage[english,russian]{babel}
\oddsidemargin=0pt
\topmargin=-2cm
\parindent=1cm

\pagestyle{empty}
\textheight=26cm
\textwidth=16cm
\flushbottom
\begin{document}
\center\section*{Математические формулы}
\begin{enumerate}
\itemПроизводная по $t$ порядка $(n-1)$ в точке $t=\tau$ терпит скачок равный единице
\[
	\frac{\partial^{n-1}G(\tau+0,\tau)}{\partial t^{n-1}}-\frac{\partial^{n-1}G(\tau-0,\tau)}{\partial t^{n-1}}=1.
\]
\itemВектор-функция $g(t)$ имеет вид
\[
	g(t)=\int\limits_a^t (b(\tau)+\Phi^{-1}e_n)f(\tau)dt+\int\limits_t^b b(\tau)f(\tau)dt.
\]
\itemФункция Грина имеет вид
\[
	G(t,\tau)=
			\begin{cases}
				
			\end{cases}
\]
\end{enumerate}
\end{document}