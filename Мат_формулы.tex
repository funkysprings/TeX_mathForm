\documentclass[11pt]{article}
\usepackage[T2A]{fontenc}
\usepackage[utf8]{inputenc}
\usepackage[english,russian]{babel}
\usepackage{amsmath}
\oddsidemargin=0pt
\topmargin=-2cm
\parindent=1cm

\pagestyle{empty}
\textheight=26cm
\textwidth=16cm
\flushbottom
\begin{document}
\center\section*{Математические формулы}
\begin{enumerate}
\itemПроизводная по $t$ порядка $(n-1)$ в точке $t=\tau$ терпит скачок равный единице
\[
	\frac{\partial^{n-1}G(\tau+0,\tau)}{\partial t^{n-1}}-\frac{\partial^{n-1}G(\tau-0,\tau)}{\partial t^{n-1}}=1.
\]
\itemВектор-функция $g(t)$ имеет вид
\[
	g(t)=\int\limits_a^t (b(\tau)+\Phi^{-1}e_n)f(\tau)dt+\int\limits_t^b b(\tau)f(\tau)dt.
\]
\itemФункция Грина имеет вид
\[
			G(t,\tau)=
			\begin{cases}
				\Phi(t)F\Phi^{-1}(\tau), & t\le\tau,\\
				\Phi(t)(I+F)\Phi^{-1}(\tau), & \tau<t,
			\end{cases}
\]
\\ где $F=-V^{-1}N\Phi(b)$.
\itemСлабосходящийся ряд
\[
	\sum\limits_{k=0}^{\infty}e^{-k(A+iB)}=
		\frac
			{e^A-\cos(B)+i\sin(B)}
			{2(\ch(A)-\cos(B))}
\]
\itemВыборочный множественный коэффициент корреляции
\[
	\widehat R_{yx}^2=1-\frac{\det R}{\left| R\right|_{00}}=
		\begin{pmatrix}
			1 & \hat r_{yx_1} & \hat r_{yx_2} \\
			\hat r_{x_1y} & 1 & \hat r_{x_1x_2} \\
			\hat r_{x_2y} & \hat r_{x_2x_1} & 1
		\end{pmatrix}
\]
\itemКоэффициент корреляции Спирмена
\[
	r^c=1-\frac
		{6\sum\limits_{i=1}^{n}d_i^2}
		{n(n^2-1)}
\]
\itemХарактеристика $\chi^2$
\[
	C=\left[ 
		\frac
			{\chi_{ct}^2}
			{n\cdot \min(m_1-1,m_2-1)}
	    \right]^{\frac 12}
\]
\itemКоэффициенты Пирсона и Чупрова
\[
	K_{\mbox{\tinyП}}=\sqrt{\frac {\varphi^2}{1+\varphi^2}} ,\quad
	K_{\mbox{\tinyЧ}}=\sqrt{\frac {\varphi^2}{\sqrt{(m_1-1)(m_2-1)}}}
\]
\itemФормула
\[
		f(x,y,\alpha, \beta) =
		 	\frac
				{\sum \limits_{n=1}^{\infty}A_n \cos \left( \frac{2 n \pi x}{\nu} \right)}
			 	{\prod \mathcal{F} {g(x,y)} } 
\]
\end{enumerate}
\end{document}